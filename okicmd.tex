% !TEX encoding = UTF-8 Unicode
% !TEX root = okicmd.tex
\RequirePackage[l2tabu, orthodox]{nag}
\documentclass[11pt, a4paper]{article}

% Margin & fonts (delete them to use a provided class style)
\usepackage[top=1truein, bottom=1truein, left=1truein, right=1truein]{geometry}
\usepackage{type1cm}
\usepackage[utf8]{inputenc}
\usepackage[T1]{fontenc}
\usepackage{lmodern}

% hyperref
\usepackage[bookmarksnumbered, pdfdisplaydoctitle, pdfusetitle, unicode]{hyperref}

% Misc
\usepackage{autonum}
\usepackage{booktabs}
\usepackage[en-US]{datetime2}

% Code execution and highlight
\usepackage{cnltx-example}
\newsourcecodeenv{cnltxexample}
\let\example\relax
\let\endexample\relax

% okicmd
\usepackage{okicmd}
\usepackage{okithm}


\title{The \pkg*{okicmd} and \pkg*{okithm} Packages}
\author{Taihei Oki}

\begin{document}
\maketitle

\section{The \pkg*{okicmd} Package}

\subsection{Alphabets}
\begin{center}
  \begin{tabular}{lc} \toprule
    \multicolumn{1}{c}{Input} & Output        \\\midrule
    \code{l}                  & $l$           \\
    \cs{ell}                  & $\ell$        \\
    \cs{epsilon}              & $\epsilon$    \\
    \cs{varepsilon}           & $\varepsilon$ \\
    \cs{phi}                  & $\phi$        \\
    \cs{varphi}               & $\varphi$     \\
    \bottomrule
  \end{tabular}
\end{center}

\subsection{Parentheses}
\begin{center}
  \begin{tabular}{lc} \toprule
    \multicolumn{1}{c}{Input}                & Output               \\\midrule
    \cs{prn}\Marg{\cs{cdot}}                 & $\prn{\cdot}$        \\
    \cs{prn}\Oarg{\cs{big}}\Marg{\cs{cdot}}  & $\prn[\big]{\cdot}$  \\
    \cs{prn}\Oarg{\cs{Big}}\Marg{\cs{cdot}}  & $\prn[\Big]{\cdot}$  \\
    \cs{prn}\Oarg{\cs{bigg}}\Marg{\cs{cdot}} & $\prn[\bigg]{\cdot}$ \\
    \cs{prn}\Oarg{\cs{Bigg}}\Marg{\cs{cdot}} & $\prn[\Bigg]{\cdot}$ \\
    \cs{curl}\Marg{\cs{cdot}}                & $\curl{\cdot}$       \\
    \cs{sqbr}\Marg{\cs{cdot}}                & $\sqbr{\cdot}$       \\
    \cs{agbr}\Marg{\cs{cdot}}                & $\agbr{\cdot}$       \\
    \cs{dbbr}\Marg{\cs{cdot}}                & $\dbbr{\cdot}$       \\
    \cs{abs}\Marg{\cs{cdot}}                 & $\abs{\cdot}$        \\
    \cs{norm}\Marg{\cs{cdot}}                & $\norm{\cdot}$       \\
    \cs{floor}\Marg{\cs{cdot}}               & $\floor{\cdot}$      \\
    \cs{ceil}\Marg{\cs{cdot}}                & $\ceil{\cdot}$       \\
    \bottomrule
  \end{tabular}
\end{center}

\subsection{Logic}
\begin{center}
  \begin{tabular}{lc} \toprule
    \multicolumn{1}{c}{Input} & Output        \\\midrule
    \cs{bigland}              & $\bigland$    \\
    \cs{biglor}               & $\biglor$     \\
    \code{a \cs{defeq} b}     & $a \defeq b$  \\
    \code{a \cs{eqdef} b}     & $b \eqdef a$  \\
    \code{P \cs{defiff} Q}    & $P \defiff Q$ \\
    \bottomrule
  \end{tabular}
\end{center}

\subsection{Sets}
\begin{center}
  \begin{tabular}{lc} \toprule
    \multicolumn{1}{c}{Input}                       & Output                   \\\midrule
    \cs{set}\Marg{a \cs{in} S}                      & $\set{a \in S}$          \\
    \cs{set}\Marg{a \cs{in} S}\Oarg{a\string^2 = 1} & $\set{a \in S}[a^2 = 1]$ \\
    \cs{intset}\Marg{n}                             & $\intset{n}$             \\
    \cs{card}\Marg{X}                               & $\card{X}$               \\
    \code{X \cs{symdif} Y}                          & $X \symdif Y$            \\
    \cs{setN}                                       & $\setN$                  \\
    \cs{setZ}                                       & $\setZ$                  \\
    \cs{setQ}                                       & $\setQ$                  \\
    \cs{setR}                                       & $\setR$                  \\
    \cs{setC}                                       & $\setC$                  \\
    \cs{setH}                                       & $\setH$                  \\
    \cs{setF}                                       & $\setF$                  \\
    \cs{setK}                                       & $\setK$                  \\
    \cs{setZp}                                      & $\setZp$                 \\
    \cs{setQp}                                      & $\setQp$                 \\
    \cs{setRp}                                      & $\setRp$                 \\
    \bottomrule
  \end{tabular}
\end{center}

\subsection{Maps}
\begin{center}
  \begin{tabular}{lc} \toprule
    \multicolumn{1}{c}{Input}             & Output               \\\midrule
    \cs{doms}\Marg{X}\Marg{Y}             & $\doms{X}{Y}$        \\
    \cs{funcdoms}\Marg{f}\Marg{X}\Marg{Y} & $\funcdoms{f}{X}{Y}$ \\
    \cs{restr}\Marg{f}\Marg{S}            & $\restr{f}{S}$       \\
    \code{\cs{id}\_K}                     & $\id_K$              \\
    \code{\cs{dom} f}                     & $\dom f$             \\
    \code{\cs{cod} f}                     & $\cod f$             \\
    \code{\cs{supp} f}                    & $\supp f$            \\
    \bottomrule
  \end{tabular}
\end{center}

\subsection{Lattices}
\begin{center}
  \begin{tabular}{lc} \toprule
    \multicolumn{1}{c}{Input} & Output      \\\midrule
    \code{x \cs{meet} y}      & $x \meet y$ \\
    \code{x \cs{join} y}      & $x \join y$ \\
    \cs{bigmeet}              & $\bigmeet$  \\
    \cs{bigjoin}              & $\bigjoin$  \\
    \bottomrule
  \end{tabular}
\end{center}

\subsection{Algebra}
\begin{center}
  \begin{tabular}{lc} \toprule
    \multicolumn{1}{c}{Input}                  & Output               \\\midrule
    \cs{Hom}\Darg{G}                           & $\Hom(G)$            \\
    \code{\cs{End} R}                          & $\End R$             \\
    \code{\cs{Aut}\_k K}                       & $\Aut_k K$           \\
    \cs{gen}\Marg{a, b}                        & $\gen{a, b}$         \\
    \cs{gen}\Marg{a, b}\Oarg{ab = e}           & $\gen{a, b}[ab = e]$ \\
    \cs{abel}\Marg{G}                          & $\abel{G}$           \\
    \cs{comm}\Marg{G}                          & $\comm{G}$           \\
    \code{\cs{sym}\_n}                         & $\sym_n$             \\
    \cs{sgn}\Darg{\cs{sigma}}                  & $\sgn(\sigma)$       \\
    \cs{mult}\Marg{R}                          & $\mult{R}$           \\
    \code{\cs{M}\_\string{m,n\string}\Darg{R}} & $\M_{m,n}(R)$        \\
    \code{\cs{GL}\_n\Darg{R}}                  & $\GL_n(R)$           \\
    \code{\cs{SL}\_n\Darg{R}}                  & $\SL_n(R)$           \\
    \cs{O}\Darg{n}                             & $\O(n)$              \\
    \cs{SO}\Darg{n}                            & $\SO(n)$             \\
    \cs{U}\Darg{n}                             & $\U(n)$              \\
    \cs{SU}\Darg{n}                            & $\SU(n)$             \\
    \bottomrule
  \end{tabular}
\end{center}

\subsection{Number Theory}
\begin{center}
  \begin{tabular}{lc} \toprule
    \multicolumn{1}{c}{Input} & Output          \\\midrule
    \code{a \cs{coprime} b}   & $a \coprime b$  \\
    \code{a \cs{divides} b}   & $a \divides b$  \\
    \code{a \cs{ndivides} b}  & $a \ndivides b$ \\
    \bottomrule
  \end{tabular}
\end{center}

\subsection{Linear Algebra}
\begin{center}
  \begin{tabular}{lc} \toprule
    \multicolumn{1}{c}{Input}                   & Output                         \\\midrule
    \code{\cs{tr} A}                            & $\tr A$                        \\
    \code{\cs{rank} A}                          & $\rank A$                      \\
    \code{\cs{trank} A}                         & $\trank A$                     \\
    \cs{diag}\Darg{a\_1, \cs{ldots}, a\_n}      & $\diag(a_1, \ldots, a_n)$      \\
    \cs{blockdiag}\Darg{A\_1, \cs{ldots}, A\_n} & $\blockdiag(A_1, \ldots, A_n)$ \\
    \cs{vectorize}\Darg{A}                      & $\vectorize(A)$                \\
    \cs{Row}\Darg{A}                            & $\Row(A)$                      \\
    \cs{Col}\Darg{A}                            & $\Col(A)$                      \\
    \cs{onevec}                                 & $\onevec$                      \\
    \cs{trsp}\Marg{A}                           & $\trsp{A}$                     \\
    \cs{adjo}\Marg{A}                           & $\adjo{A}$                     \\
    \cs{inpr}\Marg{x}\Marg{y}                   & $\inpr{x}{y}$                  \\
    \bottomrule
  \end{tabular}
\end{center}

\subsection{Analysis}
\begin{center}
  \begin{tabular}{lc} \toprule
    \multicolumn{1}{c}{Input}  & Output         \\\midrule
    \cs{e}                     & $\e$           \\
    \cs{d}                     & $\d$           \\
    \cs{dif}\Marg{f}\Marg{x}   & $\dif{f}{x}$   \\
    \cs{pdif}\Marg{f}\Marg{x}  & $\pdif{f}{x}$  \\
    \cs{ddif}\Marg{f}\Marg{x}  & $\ddif{f}{x}$  \\
    \cs{dpdif}\Marg{f}\Marg{x} & $\dpdif{f}{x}$ \\
    \bottomrule
  \end{tabular}
\end{center}

\subsection{Complex Analysis}
\begin{center}
  \begin{tabular}{lc} \toprule
    \multicolumn{1}{c}{Input}                      & Output            \\\midrule
    \cs{i}                                         & $\i$              \\
    \code{\cs{Re} z}                               & $\Re z$           \\
    \code{\cs{Im} z}                               & $\Im z$           \\
    \code{\cs{Arg} z}                              & $\Arg z$          \\
    \code{\cs{Loc} z}                              & $\Log z$          \\
    \code{\cs{Sin} z}                              & $\Sin z$          \\
    \code{\cs{Cos} z}                              & $\Cos z$          \\
    \code{\cs{Tan} z}                              & $\Tan z$          \\
    \code{\cs{Res}\_\string{z=0\string} f\Darg{z}} & $\Res_{z=0} f(z)$ \\
    \bottomrule
  \end{tabular}
\end{center}

\subsection{Optimization}
\begin{center}
  \begin{tabular}{lc} \toprule
    \multicolumn{1}{c}{Input}                                 & Output                   \\\midrule
    \code{\cs{argmin}\_\string{x \cs{in} S\string} f\Darg{x}} & $\argmin_{x \in S} f(x)$ \\
    \code{\cs{argmax}\_\string{x \cs{in} S\string} f\Darg{x}} & $\argmax_{x \in S} f(x)$ \\
    \cs{Order}\Darg{n}                                        & $\Order(n)$              \\
    \cs{order}\Darg{n}                                        & $\order(n)$              \\
    \bottomrule
  \end{tabular}
\end{center}

\section{The \pkg*{okithm} Package}
\subsection{Theorems}

If the language is set to Japanese like by \cs{usepackage}\Oarg{\keyis-{main}{japanese}}\Marg{\pkg{babel}}, \pkg{okithm} will translate all the environment titles (Theorem, Definition, etc.) into Japanese.
You can disable theorems by giving the option \option{notheorem} to \pkg{okicmd}.

\begin{cnltxexample}
  \begin{theorem}[Awesome theorem]
    The square root $\sqrt{2}$ of two is irrational.
  \end{theorem}

  \begin{definition}[Coprime]
    Integers $a$ and $b$ are said to be \emph{coprime} if their greatest common divisor is one.
  \end{definition}

  \begin{lemma}
    If $a$ and $b$ are coprime, so are $a^2$ and $b^2$.
  \end{lemma}

  \begin{proposition}
    If $\sqrt{2} = a/b$, then $a^2 = 2b^2$.
  \end{proposition}

  \begin{corollary}
    If $\sqrt{2} = a/b$ with $a$ and $b$ being coprime, then $a$ is even.
  \end{corollary}

  \begin{example}
    If $a = 2$ and $b = 1$, then $a$ is even but $\sqrt{2} \ne a/b$.
  \end{example}

  \begin{remark}
    Note that $a$ and $b$ must be integers.
  \end{remark}

  \begin{proof}[of Awesome theorem]
    Suppose to the contrary that $\sqrt{2} = a/b$ with coprime $a$ and $b$.
    Then both $a$ and $b$ are even, which contradicts the assumption.
  \end{proof}
\end{cnltxexample}

\subsection{Algorithms}

You can disable algorithms by setting the option \option{noalgorithm}.

\begin{cnltxexample}
  \begin{algorithmic}[1]
    \Input{$n \in \setN$}
    \Output{$n(n+1)/2$}
    \State{$s \gets 0$}
    \ForTo{$i = 1$}{$n$}
    \State{$s \gets s + i$}
    \EndFor
    \State{\Return $s$}
  \end{algorithmic}
\end{cnltxexample}

\subsection{Optimization Problems}

You can change \texttt{minimize}, \texttt{maximize} and \texttt{subject to} into \texttt{min}, \texttt{max} and \texttt{s.t.}, respectively, by setting the option \keyis-{optstyle}{short}.

\begin{cnltxexample}
  \Minimize[name={(P)}]{
    \sum_{\condit{x \in S}[x^2 = 1]} w (x) + \sum_{i=1}^n (p_i + q_i)
  }{
    S \subseteq V, \\
    p_i \ge 0 & (i = 1, \ldots, n), \\
    q_i \ge 0 & (i = 1, \ldots, n)
  }
\end{cnltxexample}

\end{document}
